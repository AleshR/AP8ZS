%----------------------------------------------------------------------------------------

\documentclass{article}
\usepackage[czech]{babel}
\usepackage[utf8]{inputenc}
\usepackage{verbatim}
\usepackage{listingsutf8}
\usepackage{graphicx}
\usepackage{hyperref}

\hypersetup{
    colorlinks=true,
    linkcolor=blue,
    filecolor=magenta,
    urlcolor=cyan,
}

\setlength\parindent{0pt} % Removes all indentation from paragraphs

\renewcommand{\labelenumi}{\alph{enumi}.} % Make numbering in the enumerate environment by letter rather than number (e.g. section 6)
\usepackage{xcolor}
\lstset{
    language=bash, %% Troque para PHP, C, Java, etc... bash é o padrão
    basicstyle=\ttfamily\small,
    numberstyle=\footnotesize,
    numbers=left,
    backgroundcolor=\color{gray!10},
    frame=single,
    tabsize=2,
    rulecolor=\color{black!30},
    title=\lstname,
    escapeinside={\%*}{*)},
    breaklines=true,
    breakatwhitespace=true,
    framextopmargin=2pt,
    framexbottommargin=2pt,
    extendedchars=false,
    inputencoding=utf8
}

%\usepackage{times} % Uncomment to use the Times New Roman font

%----------------------------------------------------------------------------------------
%	DOCUMENT INFORMATION
%----------------------------------------------------------------------------------------

\title{Zpracování signálů} % Title

\author{Bc. Aleš Ryška} % Author name

\date{\today} % Date for the report

\begin{document}

\maketitle % Insert the title, author and date

% If you wish to include an abstract, uncomment the lines below
% \begin{abstract}
% Abstract text
% \end{abstract}

%----------------------------------------------------------------------------------------
%	SECTION 1
%----------------------------------------------------------------------------------------

\section{Zadání}
Vypočítejte odezvu systému \textbf{y[n]} na vstupní diskrétní posloupnost \textbf{x[n]}, když znáte impulsní odezvu systému \textbf{h[n]} (tzv. konvoluční jádro).\\

Uvažujte sekvenci:
$$x = [0,1,2,3,2,1]$$
Uživatel si vybere z těchto impulzních odezev \textbf{h[k]}:\\
\begin{itemize}
	\item h = [1]
	\item h = [0,0,0,0,1]
	\item h = [-1,1]
	\item h = [0,0,0,0,-1,1]
	\item h = [3]
	\item h = [0,0,0,0,3]
	\item h = [0,0,1,2,3]
\end{itemize}

Naprogramujte vlastní algoritmus výpočtu diskrétní lineární konvoluce a ověřte s vestavěnou funkcí matlabu „conv“.\\
\textbf{Použijte tyto sekvence:}\\
\begin{itemize}
	\item \textbf{x = [0 0 1 2 3 4 4 4 3 3 2 2 1 1]}
	\item \textbf{h = [0 0 0 2 1]}
\end{itemize}
\section{Vypracování}
\includegraphics[scale=0.75]{../assets/img1.png}
\includegraphics[scale=0.75]{../assets/img2.png}
\section{Kód}
%--------------------------------------------------------------------------------------------------
%	CODE
%--------------------------------------------------------------------------------------------------
\lstinputlisting[language=Octave]{../code/convolution.m}
\href{https://github.com/AleshR/AP8ZS}{Odkaz na kompetní repozitář se cvičeními}
\end{document}

