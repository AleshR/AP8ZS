%----------------------------------------------------------------------------------------

\documentclass{article}
\usepackage[czech]{babel}
\usepackage[utf8]{inputenc}
\usepackage{verbatim}
\usepackage{listings}
\setlength\parindent{0pt} % Removes all indentation from paragraphs

\renewcommand{\labelenumi}{\alph{enumi}.} % Make numbering in the enumerate environment by letter rather than number (e.g. section 6)

%\usepackage{times} % Uncomment to use the Times New Roman font

%----------------------------------------------------------------------------------------
%	DOCUMENT INFORMATION
%----------------------------------------------------------------------------------------

\title{Zpracování signálů} % Title

\author{Bc. Aleš Ryška} % Author name

\date{\today} % Date for the report

\begin{document}

\maketitle % Insert the title, author and date

% If you wish to include an abstract, uncomment the lines below
% \begin{abstract}
% Abstract text
% \end{abstract}

%----------------------------------------------------------------------------------------
%	SECTION 1
%----------------------------------------------------------------------------------------

\section{Zadání}
Vygenerujte navzorkovaný sinusový, obdélníkový a pilový napěťový signál o daných vlastnostech (stejnosměrná složka,amplituda, frekvence signálu, počáteční fáze, vzorkovací frekvence, délka signálu).\\
Všechny tyto parametry se budou dát nastavit v samostatných proměnných!\\
Po spuštění byste měli dosáhnout vzhledu, který vidíte na obrázku níže. Vytvořte protokol v MS Word (doc,docx), vložte obrázek průběhu signálů včetně textu funkčního zdrojového kódu (MATLAB nebo Mathematica) a vložte Váš individuální komentář se závěry.
\section{Vypracování}

\section{Závěr}

\section{Kód}
%--------------------------------------------------------------------------------------------------
%	CODE
%--------------------------------------------------------------------------------------------------



\end{document}
